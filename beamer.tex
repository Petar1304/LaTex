\documentclass{beamer}
% \documentclass[handout]{beamer} % bez animacija

% da se vise strana nadje na jednoj (za stampanje)
% \usepackage{pgfpages}
% \pgfpagesuselayout{4 on 1}[a4paper, border shrink=2mm]

\title[prezentacija]{\Huge{Prezentacija}} % [prezentacija] running title na svakoj strani
\author{Petar Kovacevic}

% moze ali ne mora
\usetheme{default} % {Copenhagen} {Goettingen}
\setbeamercovered{transparent} % {invisible} % ovo je vazno za dinamiku
\usecolortheme{default} % {seahorse} {rose} {wolverine} {infolines}

% \usefonttheme{serif}

% packages
% \usepackage[utf8]{inputenc}
% \usepackage[T1, T2A]{fontenc}
% \usepackage[serbian]{babel}
% \usepackage{datetime}


\begin{document}

\begin{frame}
	\titlepage

\end{frame}


\section{Itemized list}

\begin{frame}{\textbf{nabrajanje}}
	\begin{itemize}
		\item  \href{http://tnt.etf.rs/~oe2em/}{http://tnt.etf.rs/~oe2em/}
		\item  druga stavka
		\item  treca stavka
		\item  jos jedna prva stavka

	\end{itemize}
\end{frame}


\begin{frame}[fragile]{code}
	\begin{verbatim}	
		public class Test{
			public static void main(String[] args) {
				System.out.println("Hello World...");
			}
		}
	\end{verbatim}
\end{frame}


\begin{frame}{\textbf{malo matematike}}
	Pitagorina teorema glasi
	\pause
	\bigskip
	\begin{equation}	
		c^2 = a^2 + b^2
	\end{equation}
	\pause

	znate li dokaz?
\end{frame}


\begin{frame}{malo slika}
	\begin{figure}
		\centering
		\includegraphics[scale=0.3]{sin.png}
		\caption{Funkcija $\sin \left( x \right)$}
		\label{slika:sinus}
	\end{figure}
\end{frame}


\section{boje}

\begin{frame}{textbf{boje}}
	\begin{Large}
	\textcolor{red}{crvena}\\
	\textcolor{blue}{plava}\\
	\textcolor{red}{plave}\\
	\textcolor{yellow}{zuta}\\
	\textcolor{cyan}{cyan}\\
	\end{Large}
\end{frame}

\beamertemplatesolidbackgroundcolor{teal!20} % 20 teal 80 white
\begin{frame}{\textcolor{teal}{\textbf{bojena pozadina}}}
	\begin{Huge}
		\textcolor{teal}{sa tamnim slovima}
	\end{Huge}
\end{frame}
\beamertemplatesolidbackgroundcolor{white}


\section{dve kolone}

\begin{frame}{\textbf{dve kolone}}
	\begin{columns}
		\column{5cm}
			gore levo \\
			dole levo \\
		\column{5cm}
			gore desno \\
			dole desno \\
	\end{columns}

\end{frame}


\end{document}
