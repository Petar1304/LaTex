\documentclass{article}

\author{Petar Kovacevic}
\title{Praktikum iz softverskih alata u elektronici}


\begin{document}

\maketitle

\tableofcontents

\bigskip

Proba za \LaTeX...

Novi red...

Novi red za dva reda...

\section{C++}
\begin{verbatim}
#include <iostream>
int main() {
	std::cout << "Hello World" << std::endl;
}

\end{verbatim}

\section{Java}
\begin{verbatim}
package test;

public class Test{
	
	public static void main(String[] args) {
		System.out.println("Hello World");
	}
}
\end{verbatim}

Liste:
\begin{itemize}
	\item prva tacka
	\item druga tacka
	\item treca tacka
\end{itemize}

Numerisana lista:
\begin{enumerate}
	\item prvi red
	\item drugi red
	\item treci red
\end{enumerate}

\section{Velicina slova}
\Huge Slova\\
\huge Slova\\
\LARGE Slova\\
\Large Slova\\
\large Slova\\
\normalsize Slova\\
\small Slova\\
\footnotesize Slova\\
\tiny Slova
\normalsize

TopLeft \hfill TopRight
\vfill
BottomLeft \hfill BottomRight

\section{Tabela:}
\begin{tabular} {|l|l|c|} % l->left c->center
\hline
		1 & 2 & 4 \\
		\hline
		5 & 6 & 7 \\
		\hline 
		8 & 9 & 10 \\
		\hline

\end{tabular}

Struktura dokumenta za tip article:
\section{Section}
\subsection{SubSection}
\subsubsection{SubSubSection}
\paragraph{Paragraph}
\subparagraph{SubParagraph}


\section*{Section bez broja}

\begin{description}
	\item[Prva rec] je prva rec
	\item[Druga rec] druga rec
	\item[Treca rec] third word
\end{description}

\end{document}
