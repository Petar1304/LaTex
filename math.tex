\documentclass{article}

\title{Math}
\author{Petar Kovacevic}

\usepackage{amssymb}
\usepackage{amsmath}

\begin{document}

\maketitle

\bigskip

Numerisana jednacina:

\begin{equation} \label{prva}
	\hat H \psi = E \psi
\end{equation}

Nenumerisana jednacina:

\begin{equation}
		i_{123} = i_1 + i_2 + i_3  \label{druga}
\end{equation}

Jednacina u tektu $i_{123} = i_1 + i_2 + i_3$.

\clearpage

Leonard Ojler je mnogo voleo jednacinu:
\[ 
		e^{i\pi} + 1 = 0 
\]

Sada se treba pozvati na jednacinu (\ref{prva}) i jednacinu(\ref{druga}).

Moglo se i pozvati jednacina \ref{prva} i \eqref{druga}.\\
Dobra praksa je da se u \verb+\label+ pise i tip.

\[
	\sqrt[4]{x} = \alpha
\]

Razlomci: $\frac{1}{2}$ i display frac: $\dfrac{a+b}{2}$.

\[
	 \left( x + 1 \right)
\]

Matrice sa jednacinama:
\[
		\boxed{
		\operatorname{h}(x) = \left \{ 
		\begin{array}{lll}
				0             & \text{if}        &    x < 0 \\
				\frac{1}{2}   & \text{if}        &    x = 0 \\
				1             & \text{if}        &    x > 0
		\end{array}
	\right.
	}
\]

Napon:
\[
	\boxed{
		V = 10 \, \text{V}
	}
\]

Matrica i determinanta:
bmatrix:
\[
	\det \begin{bmatrix} 
		a & b \\
		c & d 
	\end{bmatrix}
	 = ab - bc 
\]

matrix:
\[
	\begin{matrix}
		a & b\\
			c & d
	\end{matrix}
\]

pmatrix:
\[
	\begin{pmatrix}
			a & b \\
			c & d
	\end{pmatrix}
\]

\end{document}
